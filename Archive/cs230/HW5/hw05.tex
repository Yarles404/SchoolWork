\documentstyle[12pt]{article}
\setlength{\oddsidemargin}{12pt}
\setlength{\textwidth}{6.5in}
\setlength{\textheight}{9in}
\pagestyle{empty}
\setlength{\parskip}{7pt plus 2pt minus 2pt}

\begin{document}

    \begin{center}
    {{\large CS 230 : Discrete Computational Structures}}
        \\

        {\bf Spring Semester, 2021}\\

        {\sc Homework Assignment \#5}\\
        {\bf Due Date:}  Monday, March 15
    \end{center}

    \noindent {\bf Suggested Reading:} Rosen Sections 9.1 and 9.5; Lehman et al. Chapter 10.5, 10.6 and 10.10

    For the problems below, explain your answers and show your reasoning.

    \begin{enumerate}

        \item {\bf [10 Pts]} For each of these relations decide whether it is reflexive, anti-reflexive, symmetric, anti-symmetric and transitive. Justify your answers. $R_1$ and $R_2$ are over the set of real numbers.

        \begin{enumerate}

            \item $(x,y) \in R_1$ if and only if $xy \geq 0$ \\
                Reflexive. $a \in R$, $a * a \geq 0$. \\
                Not Anti-Reflexive. it's reflexive. \\
                Symmetric. Communitative property of multiplication, $a, b \in R$, $a * b = b * a$, ab is in the relation if and only if its equivalent expression ba is in the relation. \\
                Not Anti-Symmetric. it's Symmetric \\
                Transitive. for $xy \geq 0$, x and y must be the same sign. for $z \in R$, $yz \geq 0$, y and z must be the same sign. Therefore, x and z have the same sign, so $xz \geq 0$

            \item $(x,y) \in R_2$ if and only if $x = 2y$ \\
                Not Reflexive. $x \neq 2y$ if $x = y \land x, y \neq 0$ \\
                Not Anti-Reflexive. Consider $x = y = 0$ \\
                Not Symmetric. Consider x = 6 and y = 3. $(6, 3) \in R_2$, but $(3, 6) \in R_2$ \\
                Anti-Symmetric. $2x = y$ and $2y = x$ cannot be both be true, that is, a non-zero number's half cannot be equal to it's double \\
                Not Transitive. Suppose a = 4, b = 2, and c = 1. (a, b) and (b, c) is in the relation. However, (a, c) gives 4 = 2(1).

        \end{enumerate}

        \item {\bf [8 Pts]} Let $R_3$ be the relation on $\cal Z^+ \times Z^+$ where
        $((a,b), (c,d)) \in R_3$ if and only if $ad = bc$.

        \begin{enumerate}

            \item Prove that $R_3$ is an equivalence relation.
                \begin{enumerate}
                    \item Reflexive: $ab = ba \to (a,b) R (a,b)$ for all $(a,b) \in Z \times Z$
                    \item Symmetric: $(a,b)R(c,d) \to ad = bc \to cb = da \to (c,d)R(a,b)$
                    \item Transitive: Prove $(a,b)R(c,d) \land (c,d)R(e,f) \to (a,b)R(e,f)$ \\
                            $ad = bc \land cf = de$, $c = \frac{de}{f}$ \\
                            $ad = b*\frac{de}{f}$ \\
                            $af = b*e) \to (a,b)R(e,f)$
                \end{enumerate}
            \item Define a function $f$ such that $f(a,b) = f(c,d)$ if and only if $((a,b), (c,d)) \in R_3$.
                $f(x,y) = \frac{x}{y}$
            \item Define the equivalence class containing $(1,1)$. \\
                $[(1,1)]_R = \{ (a,b)\; |\; a = b, (a,b) \in Z \times Z \}$
            \item Describe the equivalence classes. How many classes are there and how many elements in each class? \\
                $\forall (a,b) \in Z \times Z$, $[(a,b)]_R = \{ (c,d)\; |\; ad = bc, (c,d) \in Z \times Z \}$ \\
                There's a countably infinite no. of classes, because $(a,b) \in Z \times Z$ is countably infinite. \\
                Each class has an infinitely countable no. of elements, because $(c,d) \in Z \times Z$ is countably infinite.

        \end{enumerate}

        \item {\bf [8 Pts]} Are these relations on the set of 5 digit numbers equivalence relations? If so, prove the properties satisfied, describe the equivalence classes and describe a new equivalence relation which is a refinement of the relation given. If not, describe which properties are violated.

        \begin{enumerate}

            \item $(a,b) \in R_4$ if and only if $a$ and $b$ start with the same two digits \\
                \begin{enumerate}
                    \item Reflexive: A number's first two digits are equal to it's doppleganger's two digits.
                    \item Symmetric: When the first two digits match, comparing them in reverse order doesn't change the match
                    \item Transitive: If number A's 2 digits match B's, and B matches C's, then A's digits = B's digits = C's digits.
                    \item Each class contains 1000 elements, with ab000-ab999,
                    \item There are 100 equivalence classes, where the first two digits are 00-99
                    \item A refinement would be all five digits numbers with the same 3 first digits, with 1000 equivlance classes 000-999 and 100 elements per class
                \end{enumerate}
            \item $(a,b) \in R_5$ if and only if $a$ and $b$ have the same $k$th digit, where $k$ is a number from 1 to 5
                \begin{enumerate}
                    \item Reflexive: a five digit number has the same digits as it's doppleganger. this is true for all k locations
                    \item Symmetric: Two numbers with the same kth digit compared in reversed order will not change the matching digit at location k
                    \item Transitive: If A's kth digit matches B's digit, and B's matches C's, then kth digit of A = B = C
                    \item There are five equivalance classes, for k 1-5
                    \item Each class has 1000*10 elements, where 10 is number of possible values for the kth digit, and 1000 are the number of possible combinations of the four other digits.
                    \item This relation could be refined by defining the set as all five digit numbers with the same kth and lth digits, where k and l are 1-5. This would have $\frac{5*4}{2}$ equivalence classes, and each class would have 100 elements.
                \end{enumerate}
        \end{enumerate}

        \item {\bf [12 Pts]} Prove that these relations on the set of all functions from $\cal Z$ to $\cal Z$ are equivalence relations. Describe the equivalence classes.

        \begin{enumerate}

            \item $R_6 = \{(f,g) \mid f(0) = g(0) \; {\rm and} \; f(1) = g(1)\}$
                \begin{enumerate}
                    \item Reflexive: If f, g are the same function in the relation, then f(0) = g(0), f(1) = g(1)
                    \item Symmetric: For f, g in the relation, If f(0) = g(0), f(1) = g(1), then g(0) = f(0), g(1) = f(1)
                    \item Transitive: for functions f, g, h in the relation, f(0) = g(0) = h(0), f(1) = g(1) = h(1)
                    \item There are an uncountably infinite number of piecewise functions where for arbitrary constant $C \in Z$, set f(0) = C, g(0) = C, and $D \in Z$ f(1) = D, g(1) = D.  Each function has a countably infinite number of elements because they will only vary by some number of constants. $Z \times Z \times Z \times ...$ is countably infinite.
                \end{enumerate}
            \item $R_7 = \{(f,g) \mid \exists C \in {\cal Z}, \forall x \in {\cal Z}, f(x) - g(x) = C\}$
                \begin{enumerate}
                    \item Reflexive: f(x) - f(x) = 0. $0 \in Z$
                    \item Symmetric: f(x) - g(x) = C, g(x) - f(x) = -C. $C, -C \in Z$
                    \item Transitive: Because we must consider $\forall x \in Z$, the domain of f and g must be Z. All Z have some output in Z, and any output $a, b \in Z$, $a - b \in Z$ All differences between the output of 3 functions with domain Z will exist in Z.
                    \item There's an uncountably infinite number of functions with domain in Z, so there are an uncountably infinite number of classes. Each function has a countably infinite number of elements because they will only vary by up to an infinite number of constants. $Z \times Z \times Z \times ...$ is countably infinite.
                \end{enumerate}
        \end{enumerate}
        \newpage

        \item {\bf [12 Pts]} Consider the following relations on the set of positive real numbers. One is an equivalence relation and the other is a partial order. Which is which? For the equivalence relation, describe the equivalence classes. What is the equivalence class of $2$? of $\pi$? Justify your answers.

        \begin{enumerate}

            \item $(x,y) \in R_8$ if and only if $x/y \in \cal Z$. THIS IS THE PARTIAL ORDER
                \begin{enumerate}
                    \item Reflexive: for any number $a \neq 0$, $a/a = 1 \in Z$
                    \item Anti-Symmetric: $\frac{y}{x}$ is the reciprocal of $\frac{x}{y}$. if $(x,y) \in R_8$, then the quotient is an integer. The reciprocal of an integer will never be an integer, except if x = y.
                    \item Transitive: if $\frac{x}{y}$ is an integer, then that implies y divides x. If $\frac{y}{z}$ is an integer, then z divides y. z will then also be a factor of x, so z divides x.
                \end{enumerate}

            \item $(x,y) \in R_9$ if and only if $x-y \in \cal Z$. THIS IS THE EQUIVALENCE RELATION
                \begin{enumerate}
                    \item Since (a) is the partial order, this must be the equivalence relation.
                    \item $(x, y) \in R \subset N \times N$ Countably infinite number of classes, where each class is determined by the value of x. Each class has a countably infinite number of y where x - y = an integer.
                    \item The class for 2 is all integers. 2 is an integer, and any integer minus an integer will be in Z
                    \item $[pi]_R = \{ \pi + n \;|\; n \in Z \} $
                \end{enumerate}
        \end{enumerate}


    \end{enumerate}
\end{document}

\noindent
For more practice, work on the problems from Sections 9.1 and 9.5; Lehman et al. Chapter 10.5, 10.6 and 10.10

\end{document}