\documentclass[11pt]{article}

\newcommand{\xor}{\oplus}

\title{COM S 230 HW03}
\author{Charles Yang}

\addtolength{\oddsidemargin}{-.875in}
\addtolength{\evensidemargin}{-.875in}
\addtolength{\textwidth}{1.75in}
\addtolength{\topmargin}{-.875in}
\addtolength{\textheight}{1.75in}

% Packages
\usepackage{amsmath}
\usepackage{amsfonts}
\usepackage{hyperref}

% Document
\begin{document}
    \maketitle

    \begin{enumerate}
        \item Prove ($p^3$ is odd) $\iff (p$ is odd)\\\\
            Assume $p^3$ is odd
            \begin{align*}
                p^3 &= 2n + 1, n \in \mathbb{Z}\\
                2n &= p^3 - 1\\
                2n &= (p-1)(p^2+p+1)
            \end{align*}
            $(p^2+p) = p(p+1)$ is even, because it is the product of an even and odd integer\\
            Therefore, $p^2+p+1$ is odd\\
            If $p^2+p+1$ is odd and $2n$ is even, then $p-1$ must be even for $2n = (p-1)(p^2+p+1)$ to remain consistent\\
            If $p - 1$ is even, then $p$ must be odd; QED
        \item If x, y is rational and z is irrational, prove $x+yz$ is irrational\\
            We cannot use a diret proof because of the nature of irrational numbers\\
            For proof by contradiction, assume that $x+yz$ is rational, such that
            \begin{center}
                $x = \frac{a}{b}$, $y = \frac{c}{d}$, and $\frac{a}{b}+\frac{c}{d}z = \frac{e}{f}$\\
                $z=(\frac{e}{f}-\frac{a}{b})\frac{d}{c}$
            \end{center}
            Left hand side is z, an irrational, and right hand side is sum/product of rational integers, which is rational.
            This is a contradiction. $x+yz$ cannot be rational; QED
        \item Prove that if $mn > 35$, then $m \geq 6 \lor n \geq 8$
            \begin{description}
                \item[(1)] If $\neg (m \geq 6 \lor n \geq 8)$, then $mn \leq 35$ \null\hfill Contrapositive of proposition
                \item[(2)] $\neg (m \geq 6 \lor n \geq 8)$ \null\hfill Assumed
                \item[(3)] $(m < 6 \land n < 8)$ \null\hfill De Morgan's (2)
                \item[(4)] $(m \leq 5 \land n \leq 7)$ \null\hfill implied in (3) because m \& n are integers
                \item[(5)] Because of (4), $mn \leq 35$
                \item[(6)] Contrapositive (1) of proposition is satisfied; QED
            \end{description}
        \item $fresh + soph + jun + sen = 32$. Prove $fresh \geq 5 \lor soph \geq 8 \lor jun \geq 10 \lor sen \geq 7$
            \begin{description}
                \item[(1)] $(fresh + soph + jun + sen = 32) \to (fresh \geq 5 \lor soph \geq 8 \lor jun \geq 10 \lor sen \geq 7)$ \null\hfill Hyp. 1
                \item[(2)] $\neg (fresh \geq 5 \lor soph \geq 8 \lor jun \geq 10 \lor sen \geq 7) \to (fresh + soph + jun + sen \neq 32)$ \\\null\hfill Contrapositive (1)
                \item[(3)] $\neg (fresh \geq 5 \lor soph \geq 8 \lor jun \geq 10 \lor sen \geq 7)$ \null\hfill Assumed
                \item[(4)] $fresh < 5 \land soph < 8 \land jun < 10 \land sen < 7)$ \null\hfill De Morgan's (3)
                \item[(5)] $fresh \leq 4 \land soph \leq 7 \land jun \leq 9 \land sen \leq 6)$ \null\hfill Implied in (4) because all vars are ints
                \item[(6)] $(fresh + soph + jun + sen \neq 32)$ implied by (5); all vars could be 0
                \item[(7)] (2) is satisfied; QED
            \end{description}
        \item Prove $(p \geq 3 \lor p \leq -7) \to ((p+2)^2) \geq 25$
            \begin{description}
                \item[(1)] Case 1: $p \geq 3$, add 2: $p+2 \geq 5$, square: $(p + 2)^2 \geq 25$
                \item[(2)] Case 2: $p \leq -7$, add 2: $p + 2 \leq -5$, square: $(p + 2)^2 \geq 25$
                \item[(3)] (1) \& (2); QED
            \end{description}
        \item Prove $\sqrt{5}$ is irrational
            \begin{description}
                \item[(1)] Assume $\sqrt{5}$ is rational, that is $\sqrt{5}=\frac{a}{b}$ for some integers a, b.
                \item[(2)] Square it. $5 = \frac{a^2}{b^2}$
                \item[(3)] $5b^2 = a^2$
                \item[(4)] This implies 5 is a factor of $a^2$ or $a * a$
                \item[(5)] According to \href{https://math.stackexchange.com/questions/4268/alternate-definition-of-prime-number#:~:text=A\%20prime\%20is\%20a\%20quantity,is\%20a\%20factor\%20of\%20b.}{StackExchange}, If a square is divisible by a prime, then its root is also divisible by the prime
                \item[(6)] (4) and (5) imply that 5 is a factor of a, and $a = 5x$ for some int x
                \item[(7)] $5b^2 = a^2$ and (6) implies that $5^2x^2$ is a factor of $b^2$
                \item[(8)] (7) implies that 5 is also a factor of b.
                \item[(9)] (8) contradicts (1). a and b cannot have common factors; $\sqrt{5}$ is irrational; QED
            \end{description}
    \end{enumerate}



\end{document}