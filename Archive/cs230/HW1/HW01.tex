\documentclass[12pt]{article}

\newcommand{\xor}{\oplus}

\title{COM S 230 HW01}
\author{Charles Yang}

\addtolength{\oddsidemargin}{-.875in}
\addtolength{\evensidemargin}{-.875in}
\addtolength{\textwidth}{1.75in}
\addtolength{\topmargin}{-.875in}
\addtolength{\textheight}{1.75in}

\begin{document}
    \maketitle

    \begin{enumerate}
        \item Translations from english to logic
        \begin{center}$F$: Passing the final; $A$: Attending class regularly; $P$: Pass the class.\end{center}
            \begin{description}
                \item[a)] The thing that is sufficient is first in implication \boldmath $ F \lor A \to P $
                \item[b)] if P is true, then A is necessarily true \boldmath $ P \to A $
                \item[c)] This is a straightforward translation \boldmath $ P \iff (A \land F) $
            \end{description}
        \item Implication is not associative; $((p \to q) \to r) \iff (p \to (q \to r))$ is not a tautology.
            \begin{table}[h] \begin{center} \centering
                  \begin{tabular}{ |c|c|c|c|c|c|c|c|}
                       $p$ & $q$ & $r$ & $p \to q$ & $(p \to q) \to r$ & $q \to r$ & $p \to (q \to r)$ & $((p \to q) \to r) \iff (p \to (q \to r))$ \\ \hline
                       0 & 0 & 0 & 1 & 0 & 1 & 1 & 0 \\
                       0 & 0 & 1 & 1 & 1 & 1 & 1 & 1 \\
                       0 & 1 & 0 & 1 & 0 & 0 & 1 & 0 \\
                       0 & 1 & 1 & 1 & 1 & 1 & 1 & 1 \\
                       1 & 0 & 0 & 0 & 1 & 1 & 1 & 1 \\
                       1 & 0 & 1 & 0 & 1 & 1 & 1 & 1 \\
                       1 & 1 & 0 & 1 & 0 & 0 & 0 & 1 \\
                       1 & 1 & 1 & 1 & 1 & 1 & 1 & 1
                  \end{tabular} \end{center}
            \end{table}
        \item Prove $(p \to \neg q) \land (q \to \neg r) \equiv (p \lor r) \to \neg q$
            \begin{itemize}
                \item $(\neg p \lor \neg q) \land (\neg q \lor \neg r)$ Implication 1 x 2
                \item $ \neg \neg ((\neg p \lor \neg q) \land (\neg q \lor \neg r))$ Double Negation
                \item $\neg (\neg (\neg p \lor \neg q) \lor \neg (\neg q \lor \neg r))$ De Morgan's
                \item $\neg (( p \land q) \lor ( q \land r))$ De Morgan's x 2
                \item $\neg (((p \land q) \lor q) \land ((p \land q) \lor r))$ Distribution
                \item $\neg (q \land ((p \land q) \lor r))$ Absorption
                \item $\neg (q \land ((p \lor r) \land (q \lor r)))$ Distributive
                \item $\neg (q \land (q \lor r) \land (p \lor r))$ Associative
                \item $\neg ((p \lor r) \land q)$ Absorption \& Commutative
                \item $\neg (p \lor r) \lor \neg q$ De Morgan's
                \item $(p \lor r) \to \neg q$ Implication 1
            \end{itemize}
        \item Who's lying? T is Tom's claim, and S is Sue's claim. C: $T \xor S \equiv TRUE$
            \begin{description}
                \item[1)] Suppose T is true.
                \item[2)] Then S is also true.
                \item[3)] (1) and (2) Contradicts C. T cannot be true.
                \item[4)] If Tom is the SE Major, and Sue is the CS Major, then T is false and S remains true.
                \item[5)] (4) Satisfies C. Tom is SE and Sue is CS.
            \end{description}
        \item Prove any compound proposition is logically equivalent to some proposition in CNF
            \begin{itemize}
                \item For all compound propositions, there exists $2^n$ combinations of basic propositions, where n is the number of basic propositions.
                \item For all combinations, the compound proposition is either true or false.
                \item CNF can be written as "as long as none of the variable combinations that make the proposition false are satisfied, then the compound proposition is true."
                \item Basically, $\neg ((a_1 \land a_2 \land ... \land a_n) \lor (b_1 \land b_2 \land ... \land b_n) \lor ...)$ where a, b, ... are combinations of basic propositions that make the proposition false.
                \item Applying De Morgan's, we get $\neg (a_1 \land a_2 \land ... \land a_n) \land \neg (b_1 \land b_2 \land ... \land b_n) \and ...$
                \item Again. $ (\neg a_1 \lor \neg a_2 \lor ... \lor \neg a_n) \land (\neg b_1 \lor \neg b_2 \lor ... \lor \neg b_n) \land ...$
                \item This is CNF. All conjugated clauses are the disjunction of the negation of basic propositions which make the compound proposition false.
            \end{itemize}
        \item Prove $\xor, \land, TRUE$ is functionally complete
            \begin{itemize}
                \item $\land$ is already present
                \item $(p \xor TRUE) \equiv \neg p$
                    \begin{description}
                        \item[1)] $(p \land \neg TRUE) \lor (\neg p \land TRUE)$ Definition of $\xor$
                        \item[2)] $(p \land FALSE) \lor \neg p$ Identity
                        \item[3)] $FALSE \lor \neg p$ Domination
                        \item[4)] $\neg p$ Identity
                    \end{description}
                \item $\lor$ is constructed by $\neg$ and $\land$ via De Morgan's. $\neg$ is implemented by previous proof.
                    \begin{description}
                        \item[1)] $p \lor q$
                        \item[2)] $\neg \neg (p \lor q)$ Double negation
                        \item[3)] $\neg (\neg p \land \neg q)$ De Morgan's; All necessary components for this definition are present
                    \end{description}
                \item $\neg, \land, \lor$ functionally complete, so ${\xor, \land, TRUE}$ is complete aswell.
            \end{itemize}
    \end{enumerate}
\end{document}