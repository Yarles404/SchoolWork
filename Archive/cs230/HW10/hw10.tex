\documentclass[11pt]{article}
\setlength{\oddsidemargin}{12pt}
\setlength{\textwidth}{6.5in}
\setlength{\textheight}{9in}
\pagestyle{empty}
\setlength{\parskip}{7pt plus 2pt minus 2pt}

\begin{document}

    \begin{center}
    {{\large CS 230 : Discrete Computational Structures}}
        \\

%\vspace*{1cm}

        {\bf Spring Semester, 2021}\\

        {\sc Assignment \#10} \\
        {\bf Due Date:} Monday, April 26
    \end{center}

    \noindent {\bf Suggested Reading:} Rosen Sections 6.4 - 6.5.

    These are the problems that you need to turn in. Always explain your answers and show your reasoning.
    {\bf Spend time giving a complete solution. You will be graded based on how well you explain your answers.
    Just correct answers will not be enough!}

%\vspace*{0.5cm}

    \begin{enumerate}

        \item {\bf [5 Pts]} Prove, using a combinatorial argument, that $C(m+n, 2) = C(m,2) + C(n,2) + mn$, where $m,n \geq 2$. To make your combinatorial argument, describe a problem that both the $lhs$ and $rhs$ expressions count.
        \begin{enumerate}
            \item I have two raffles: Raffle A with m participants and raffle B with n participants
            \item How many ways are there to pick 2 tickets from raffle A and B pooled? Choose 2 from m+n
            \item 3 possibilities: pick 2 only from Raffle A or Raffle B, or pick one from each bag
            \item Choose 2 from m + Choose 2 from n + Choose 1 from m and 1 from n
            \item $C(m+n, 2) = C(m,2) + C(n,2) + mn$ \null\hfill QED
        \end{enumerate}

        \item {\bf [8 Pts]} Prove, (a) using a combinatorial argument, and (b) using an algebraic proof, that $P(n,3) C(n-3, k-3) = C(n,k) P(k,3)$.
        \begin{enumerate}
            \item
            \begin{enumerate}
                \item Imagine the LHS, from n writers, is picking three writers for 1st-3rd place then selecting $k - 3$ honorable mentions in no particular order from the remaining $n - 3$ writers.
                \item LHS gurantees 3 winners and $k - 3$ honorable mentions. This results in k potential victors.
                \item Imagine RHS is picking k potential winners from n writers and then selecting 1st-3rd place from that list of $k$ writers, and the $k - 3$ not chosen are honorable mentions.
                \item This is removing 3 winners from a pool of 3 potential victors. It results in $k - 3$ honorable mentions and 3 winners.
                \item LHS and RHS are the same \null\hfill QED
            \end{enumerate}
            \item
            \begin{enumerate}
                \item $\frac{n!}{(n-3)!}*\frac{(n-3)!}{(k-3)!(n-3-(k-3))!} = \frac{n!}{(k)!(n-k)!}*\frac{k!}{(k-3)!}$
                \item $\frac{n!}{1}*\frac{1}{(k-3)!(n-k)!} = \frac{n!}{(n-k)!}*\frac{1}{(k-3)!}$
                \item $\frac{n!}{(n-k)!(k-3)!} = \frac{n!}{(n-k)!(k-3)!}$ \null\hfill QED
            \end{enumerate}
        \end{enumerate}
        \newpage

        \item {\bf [6 Pts]} A cookie shop sells 5 different kinds of cookies. How many different ways are there to choose 16 cookies if (a) you pick at least two of each? (b) you pick at least 4 oatmeal cookies and at most 4 chocolate chip cookies?
        \begin{enumerate}
            \item
            \begin{enumerate}
                \item 10 cookies already chosen. Choose 6 cookies to place in 5 bins
                \item $\frac{(6 + 5 - 1)!}{6!4!}$
                \item $\frac{10!}{6!4!}$
            \end{enumerate}
            \item
            \begin{enumerate}
                \item Count combos where 4 oatmeal already picked
                \item Subtract where 4 oatmeal already picked AND 5 or more chocolate chips picked
                \item $\frac{(12 + 5 - 1)!}{12!4!} - \frac{(7 + 5 - 1)!}{7!4!}$
                \item $\frac{16!}{12!4!} - \frac{11!}{7!4!}$
            \end{enumerate}
        \end{enumerate}

        \item {\bf [9 Pts]} How many solutions are there to the equation $x_1 + x_2 + x_3 + x_4 = 24$, where $x_i$ is a non-negative integer, for all $i$, if (a) there are no restrictions? (b) $x_1 > 1$, $x_2 > 2$, $x_3 > 3$, $x_4 > 4$? (c) $x_1 > 4$ and $x_3 < 5$?
        \begin{enumerate}
            \item
            \begin{enumerate}
                \item 24 objects in 4 bins.
                \item $\frac{(24 + 4 - 1)!}{24!3!}$
                \item $\frac{27!}{24!3!}$
            \end{enumerate}
            \item
            \begin{enumerate}
                \item 14 items already placed. choose 10 to place into 4 bins
                \item $\frac{(10 + 4 - 1)!}{10!3!}$
                \item $\frac{13!}{10!3!}$
            \end{enumerate}
            \item
            \begin{enumerate}
                \item 5 objects already placed. Compute that total
                \item Subtract the combinations where $x_1 > 4$ (5 objects placed) AND $x_3 \geq 5$ (5 objects placed)
                \item $\frac{(19 + 4 - 1)!}{19!3!} - \frac{(14 + 4 - 1)!}{14!3!}$
                \item $\frac{22!}{19!3!} - \frac{17!}{14!3!}$
            \end{enumerate}
        \end{enumerate}

        \item {\bf [10 Pts]} How many ways are there to split 30 people into three committees of 5 people each and five committees of 3 people each if (a) all eight committees have different tasks? (b) all eight committees have the same task? (c) the three 5-member committees and two of the 3-member committees are all given the same task while the remaining three 3-member committees are not given any task yet?
        \begin{enumerate}
            \item
            \begin{enumerate}
                \item 30 choose 5 * 25 choose 5 * 20 choose 5 * 15 choose 3 * 12 choose 3 * 9 choose 3 * 6 choose 3 * 3 choose 3
                \item $\frac{30!}{5!25!} * \frac{25!}{5!20!} * \frac{20!}{5!15!} * \frac{15!}{3!12!} * \frac{12!}{3!!} * \frac{9!}{3!6!} * \frac{6!}{3!3!} * \frac{3!}{3!0!}$
                \item $\frac{30!}{5!5!5!3!3!3!3!3!}$
            \end{enumerate}
            \item
            \begin{enumerate}
                \item Each pair of committees of size 5 is interchangeable and each pair of committees of size 3 is interchangeable. Division Rule on part a answer:
                \item $\frac{30!}{(5!5!5! * 3!)(3!3!3!3!3! * 5!)}$
            \end{enumerate}
            \item
            \begin{enumerate}
                \item Use division rule on committees that are indistinguishable (same/no task)
                \item $\frac{30!}{(5!5!5! * 3!)(3!3! * 2!)(3!3!3! * 3!)}$
            \end{enumerate}
        \end{enumerate}

        \item {\bf [6 Pts]} How many ways are there to pack 6 different books into 6 identical boxes with no restrictions placed on how many can go in a box (some boxes can be empty)? What if the books are identical? \\
        \begin{enumerate}
            \item
            \begin{enumerate}
                \item 6 distinguishable objects in 6 identical bins: 11 Cases enumerated by brute force: 6 | 5,1 | 4,2 | 4,1,1 | 3,3 | 3,2,1 | 3,1,1,1
                \item 6: 6 choose 6 = 1
                \item 5,1: 6 choose 5 ways = 6
                \item 4,2: 6 choose 4 = $\frac{6!}{4!2!} = 15$
                \item 4,1,1: 6 choose 4 = $\frac{6!}{4!2!} = 15$
                \item 3,3: 6 choose 3, division rule to remove duplciates = $\frac{6!}{3!3!}/2 = 10$
                \item 3,2,1: 6 choose 3 = $\frac{6!}{3!3!} = 20$
                \item 3,1,1,1: 6 choose 3 = $\frac{6!}{3!3!} = 20$
                \item 2,2,2: 6 choose 2, division rule to remove duplicates = $\frac{6!}{2!4!}/3 = 5$
                \item 2,2,1,1: 6 choose 2 = $\frac{6!}{2!4!} = 15$
                \item 2,1,1,1,1: 6 choose 2 = $\frac{6!}{2!4!} = 15$
                \item 1,1,1,1,1,1: 6 choose 1, division rule to remove duplicates = $\frac{6!}{1!5!}/6 = 1$
                \item 1 + 6 + 15 + 15 + 10 + 20 + 20 + 5 + 15 + 15 + 1 = 123 ways
            \end{enumerate}
        \end{enumerate}
        b) 6 identical objects in 6 identical bins: Just the \# of cases! 11 ways

        \item {\bf [6 Pts]} How many ways can we place 12 books on a bookcase with 5 shelves if the books are (a) indistinguishable copies (b) all distinct? Note that the position of the books on the shelves matter. \\
        a) 12 identical objects in 5 distinguishable bins: $\frac{(12 + 5 - 1)!}{12!4!} = \frac{16!}{12!4!}$ \\
        b) $12!$ ways to order the 12 books. There are 13 spots to place 4 divideres $13 * 4$ (5 bins). Every permutation of books has $13 * 4$ ways to place the dividers. $12! * 52$
    \end{enumerate}

    \noindent
    For more practice, work on the problems from Rosen Sections 6.4 - 6.5.

\end{document}

