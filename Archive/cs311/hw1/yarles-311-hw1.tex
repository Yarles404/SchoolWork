\documentclass[11pt]{article}


\usepackage{fullpage}
\usepackage{mathtools}
\usepackage{amsmath}
\usepackage{amssymb}
\date{}



\begin{document}

    \begin{center}
        \textbf{COMS 311: Homework 1}\\
        \textbf{Due: Sept 9, 11:59pm}\\
        \textbf{Total Points: 100}
    \end{center}

    \paragraph{Submission format.\ }
    Your submission should be in pdf format. Name your submission file:
    \texttt{<Your-net-id>-311-hw1.pdf}. For instance, if your netid is
    \texttt{asterix}, then your submission file will be named
    \texttt{asterix-311-hw1.pdf}.

    \paragraph{Learning outcomes.\ }
    \begin{enumerate}
        \item Determine whether or not a function is Big-O of another function
        \item Analyze asymptotic worst-case time complexity of algorithms
    \end{enumerate}

    \hrule


    \begin{enumerate}

        \item Prove or disprove the following statements. Provide a proof for
        your answers.
        \hfill (40 Points)
        \begin{enumerate}
%\item $2n^3+35n + 46 \in O(n^3)$
            \item $6n^2-41n+2 \in O(n^2)$.
            \begin{align*}
                6n^2 - 41n + 2 &\leq 6n^2 + 41n^2 + 2n^2 \\
                6n^2 - 41n + 2 &\leq 49n^2 \implies c = 49 \\
                0 &\leq 43n^2 + 41n - 2 \\
                n &= \frac{-41 \pm \sqrt{41^2 - 4(43)(-2)}}{2(43)} \\
                n &= \text{a real number, therefore there exists a } n_0 \text{. So,} \\
                6n^2 - 41n + 2 &< cn^2 \text{ for all } n > n_0
            \end{align*}
            \item $\forall a \geq 1: 2^{n}  \in  O(2^{n-a})$
            \begin{align*}
                2^n &\leq c2^{n-a} \\
                2^n &\leq 2^a*2^{n-a} \implies c = 2^{a} \text{ So, } \exists c > 0  \\
                \text{Clearly, } n_0 &= 1 \text{ satisfies the condition } \exists n_0 , \forall n \geq n_0
            \end{align*}
            \item $\forall a > 1: O(\mathtt{log}_2 n) \in O(\mathtt{log}_a n))$
            \begin{align*}
                \mathtt{log}_2 n &\leq \mathtt{log}_a n \\
                \frac{\mathtt{log}_a n}{\mathtt{log}_a 2} &\leq \mathtt{log}_a n \\
                \mathtt{log}_a n &\leq \mathtt{log}_a 2 * \mathtt{log}_a n \implies c = \mathtt{log}_a 2 \\
                \text{Clearly, } n_0 &= 1 \text{ satisfies the condition } \exists n_0 , \forall n \geq n_0
            \end{align*}
            \newpage


            \item $\forall a > 1: a^{a^{n+1}} \in O(a^{a^n})$. ???
            \begin{align*}
                \text{Assume } &\forall a > 1, \;\exists c > 1, \;\exists n_0 > 0, \;\forall n > n_0, \; a^{a^{n+1}} \leq ca^{a^n} \\
                a^{a^{n+1}} &\leq ca^{a^{n}} \\
                \frac{a^{a^{n+1}}}{a^{a^{n}}} &\leq c \\
                a^{a^{n+1}-a^{n}} &\leq c \\
                a^{a*a^{n}-a^{n}} &\leq c \\
                a^{(a - 1)*a^{n}} &\leq c \\
                &\text{LHS will outgrow c for all a and n. Disproven by Contradiction}
            \end{align*}
            \item If $f_1(n) \in O(g_1(n))$ and $f_2(n) \in O(g_2(n))$, then
            $f_1(n) + f_2(n) \in O(g_1(n) + g_2(n))$.
            \begin{align*}
                \text{Assume the following:}& \\
                \exists c_0 > 0, \; \exists n_{00} > 0, \; \forall n \geq n_{00}, \; f_1 &\leq c_0 g_1 \\
                \exists c_1 > 0, \; \exists n_{10} > 0, \; \forall n \geq n_{10}, \; f_2 &\leq c_1 g_2 \\
                \text{Therefore,}                             &\\
                \exists c_0, c_1 > 0, \; \exists n_{00}, n_{10} > 0, \; f_1 + f_2 &\leq c_0 g_1 + c_1 g_2 \\
                \text{Since LHS is less than or equal to, modifying} & \\
                \text{RHS to be larger maintains the truth of the statement. So,} & \\
                \exists c_2 = c_0 + c_1 > 0, \; f_1 + f_2 &\leq c_2 g_1 + c_2 g_2 \\
                \exists c_2 = c_0 + c_1 > 0, \; \exists n_{20} > 0, \; \forall n \geq n_{20} \; f_1 + f_2 &\leq c_2 (g_1 + g_2)
            \end{align*}


        \end{enumerate}



        \item Derive the runtime of the following as a function of $n$ and
        determine its Big-O upper bound. You must show the derivation of
        the end result. Simply stating the final answer without any
        derivation steps will result in 0 points.  \hfill (60 Points)
        \begin{enumerate}

            \item
            \begin{verbatim}
for (i = 1; i < n-8; i++) {
    for (j = i; j < i+8; j = j++) {
        <some-constant number of atomic/elementary operations>
}}
            \end{verbatim}
            \begin{align*}
                \sum_{i=1}^{n - 9} \sum_{j = i}^{i + 7} &c \\
                \sum_{i=1}^{n - 9} \sum_{j = 1}^{8} &c \\
                \sum_{i=1}^{n - 9} 8&c = (n - 9)8c \in O(n)
            \end{align*}

            \item
            \begin{verbatim}
for i in the range [1, n] {
   for j in the range [i, n] {
       for k in the range [1, j-i] {
        <some-constant number of atomic/elementary operations>
}}}
            \end{verbatim}
            \begin{align*}
                      \sum_{i=1}^{n} \sum_{j = i}^{n} \sum_{k = 1}^{j - i} &c \\
                      \sum_{i=1}^{n} \sum_{j = i}^{n} (j - i) &c \\
                      \sum_{i=1}^{n} (\sum_{j = i}^{n} j - \sum_{j = i}^{n} i) &c \\
                      \sum_{i=1}^{n} ((\sum_{j = 1}^{n} j - \sum_{j = 1}^{i} j) - n*i) &c \\
                      \sum_{i=1}^{n} ((\frac{n(n + 1)}{2} - \frac{i(i + 1)}{2} - n*i) &c) \\
                       (\sum_{i=1}^{n} \frac{n(n + 1)}{2} - \sum_{i=1}^{n} \frac{i(i + 1)}{2} - \sum_{i=1}^{n} n*i) &c \\
                       ( \frac{n^2(n + 1)}{2} - \frac{1}{2} (\sum_{i=1}^{n} i^2 + \sum_{i=1}^{n}i ) - \sum_{i=1}^{n} n*i) &c \\
                       ( \frac{n^2(n + 1)}{2} - \frac{n(n + 1)(2n + 1)}{12} + \frac{n(n + 1)}{4} - \frac{n^2(n + 1)}{2}) &c \in O(n^3)
            \end{align*}

            \item
            \begin{verbatim}
x = pow(2, n);
i = 1;
while i <= x {
  for j in range [1, i] {
        <some-constant number of atomic/elementary operations>
  }
  i = i * 2
}
            \end{verbatim}
            Assume that \texttt{pow(2, n)} (i.e., $2^n$)  is computed \emph{magically} in
            constant time.

            \begin{align*}
                &\text{The while loop will run } n + 1 \text{ times because } x = 2^n \\
                &c\sum_{i = 0}^{i = n} \sum_{j = 1}^{2^i} \\
                &c\sum_{i = 0}^{i = n} 2^i \\
                &c(2^{n + 1} - 1) \in O(2^n)
            \end{align*}

        \end{enumerate}


        \item \textbf{Extra Credit} Derive the runtime of the following in
        terms of $n$ and determine its Big-O upper bound. You must show
        the derivation of the end result. Simply stating the final answer
        without any derivation steps will result in 0 points.
        \hfill (10pts)

        \begin{verbatim}
function myf(integer a, integer n) {
    integer a1;
    while n >= 0 {
       if n==0 then return 1;
       a1 = myf(a, n/2);       
       if n is even then  {
           return a1 * a1;
       }
       else {
            return a * a1 * a1;   
       }
       n = n/2;
    }     
}
        \end{verbatim}
        Assume that $n/2 = 0$, when $n<2$. \\
        See next page
        \newpage




        The code is gross with unreachable lines. Here's the simplified and equivalent function:

        \begin{verbatim}
function myf(integer a, integer n) {
    integer a1;
    if n==0 then return 1;
    a1 = myf(a, n/2);
    if n is even then  {
        return a1 * a1;
    }
    else {
        return a * a1 * a1;
    }
}
        \end{verbatim}
        There is only one recursive call per function call, so there will be no branching. \\
        Each function call reduces interger $n$ to the previous $n / 2$ \\
        $n,\; n/2,\; n/4,\;...,\; n/2^k$ where $n/2^k \geq 1$, so $n \geq 2^k \implies k \leq log_2 n$ \\
        myf $\in O(log_2 n)$

    \end{enumerate}
\end{document}


